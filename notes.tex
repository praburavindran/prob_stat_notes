% LaTeX

\documentclass[11pt]{amsart} 

\usepackage{amsthm,amsmath}
\usepackage{amssymb}
\usepackage{enumerate}
\usepackage{pstricks}
\usepackage[utf8]{inputenc} %unicode support
\usepackage{graphicx}
\let\psgrid\relax
\psset{gridcolor = gray, subgridcolor = gray}

\theoremstyle{definition} \newtheorem{thm}{Theorem} \theoremstyle{plain}
\theoremstyle{definition} \newtheorem*{nnthm}{Theorem} \theoremstyle{plain}
\theoremstyle{definition} \newtheorem{lem}{Lemma} \theoremstyle{plain}
\theoremstyle{definition} \newtheorem*{nnlem}{Lemma} \theoremstyle{plain}
\theoremstyle{definition} \newtheorem{prf}{Proof} \theoremstyle{plain}
\theoremstyle{definition} \newtheorem*{nnprf}{Proof} \theoremstyle{plain}
\theoremstyle{definition} \newtheorem{eg}{Example} \theoremstyle{plain}
\theoremstyle{definition} \newtheorem*{sol}{Solution} \theoremstyle{plain}

\newcommand{\setcomp}[1]{{#1}^{\mathsf{c}}}
\newcommand{\prob}[1]{\mathbf{P}(#1)}
\newcommand{\twocup}[2]{{#1} \,\cup\, {#2}}
\newcommand{\threecup}[3]{{#1} \,\cup\, {#2} \,\cup\, {#3}}
\newcommand{\fourcup}[4]{{#1} \,\cup\, {#2} \,\cup\, {#3} \,\cup\, {#4}}
\newcommand{\twocap}[2]{{#1} \,\cap\, {#2}}
\newcommand{\threecap}[3]{{#1} \,\cap\, {#2} \,\cap\, {#3}}
\newcommand{\fourcap}[4]{{#1} \,\cap\, {#2} \,\cap\, {#3} \,\cap\, {#4}}
\newcommand{\sngltn}[1]{\{ {#1} \}}

%\textwidth = 460pt 
%\textheight = 8in 
%\hoffset=-54pt \voffset=-40pt

% SIDE MARGINS:
\oddsidemargin 0in \evensidemargin 0in

% VERTICAL SPACING:
%\topmargin -.15in
\topmargin -.4in
\headheight 0in \headsep 0.0in
%\footheight 0.5in
\footskip 1.5in

\pagestyle{plain}
%\pagenumbering{}

% DIMENSION OF TEXT:
\textheight 9.0in \textwidth 6.8in
%

%\textwidth = 470pt
%\textheight = 700pt
%%\topmargin = 0pt
%%\oddsidemargin = 0pt
%\hoffset = -60pt
%\voffset = -50pt

%\input epsf \def\epsfsize#1#2{0.4#1\relax} \def\nl{\hfil\break}

%\renewcommand{\baselinestretch}{1.2}
%\def\Indent{\hskip .2in}
%Sri Ramajayam
%Jai Hanuman

\title[]{Probability Law}

\author[]{Prabu Ravindran}

\begin{document}
\maketitle
\thispagestyle{empty}

\section{Sample space}
The \emph{set} of possible outcomes for an experiment is called the 
\textbf{sample space}. It is denoted as $\Omega$. The requirements to keep in 
mind when defining a sample space are:
\begin{itemize}
\item The elements of the set must be mutually exclusive. 
\item The elements of the set must be exhaustive.
\item The elements must chosen at the right granularity. This usually 
    depends on the experiment and the questions we are trying to answer. 
\end{itemize}

\begin{eg} 
For the experiment of flipping a coin, and for 
each one of the following choices, determine whether we have a legitimate 
sample space:
\begin{enumerate}
\item
$\Omega = \{\mbox{Heads and it is raining, 
                   Heads and it is not raining, 
                   Tails}\}$
\item
$\Omega = \{\mbox{Heads and it is raining, 
                  Tails and it is not raining, 
                  Tails}\}$ 
\end{enumerate}
\end{eg}
\begin{sol}
\hfill
\begin{enumerate}
\item
It is a sample space. The events are mutually exclusive and exhaustive.
\item
It is \emph{not} a sample space as the events are not mutually exclusive. 
\end{enumerate}
\end{sol} 

\section{Examples of sample space}
Example of a discrete and finite sample space is the experiment where we roll 
two tetrahedral die. We can sometimes think of an experiment as a sequential 
process or one that is made of stages. This leads to a tree based 
conceptualization of the experiment's sample space. For example, we can think 
of the two tetrahedral die experiment as rolling the first die followed by the 
second die to get the ordered pairs in the sample space. The sample space for 
this example is:
\begin{equation*}
\Omega = \{ (i, j) \mid i, j \in \{ 1, 2, 3, 4 \} \}.
\end{equation*}

Example of a continuous sample space is the experiment of throwing a dart on a 
unit square target where the position of the dart can be measured with infinite precision. 
\begin{equation*}
\Omega = \{ (x, y) \mid 0 \leq x, y \leq 1\}.
\end{equation*}

\section{Probability axioms}
The first ingredient in specifying a probability model is the sample space. The 
second ingredient is the probability law. The probability law tells how 
(relatively) likely each outcome is. This point of view runs into some issues 
especially when dealing with continuous sample spaces, where the probability 
of getting a particular outcome (with infinite precision) is zero. In order to 
overcome this problem probabilities are assigned to subsets of the sample space. 
Subsets of the sample space are called \textbf{events}. For the probability law 
to be legitimate it must adhere to the following \textbf{probability axioms}:
\begin{itemize}
\item Non-negativity: $\prob{A} \geq 0$, where $A$ is an event.
\item Normalization: $\prob{\Omega} = 1$.
\item (Finite) Additivity: If $A$ and $B$ represent events, such that 
$\twocap{A}{B} = \emptyset $, then 
$\prob{\twocup{A}{B}} = \prob{A} + \prob{B}$.
\end{itemize}
  

\section{Consequences of probability axioms}
\noindent The following are the consequences of the probability axioms:

\begin{lem}
$\prob{A} + \prob{\setcomp{A}} = 1$.
\end{lem}
\begin{nnprf}
The outcome of an experiment must either belong to a set $A$ or 
not. Thus we have: $\twocup{A}{\setcomp{A}} = \Omega$. We can also see that 
$A$ and $\setcomp{A}$ are disjoint sets. Thus using the additivity axiom we 
get $\prob{A} + \prob{\setcomp{A}} = 1$.
\end{nnprf}

\begin{lem}
$\prob{A} \leq  1$.
\end{lem}
\begin{nnprf}
From lemma above we have $\prob{A} + \prob{\setcomp{A}} = 1$. This implies that 
$\prob{A} = 1 - \prob{\setcomp{A}}$. Since probabilities as nonnegative we have 
$\prob{A} \leq 1$. 
\end{nnprf}

\begin{lem}
$\prob{\emptyset} = 0$.
\end{lem}
\begin{nnprf}
The sum of the probabilities of an event and its complement is one, from lemma 
above. Thus we have $\prob{\Omega} + \prob{\setcomp{\Omega}} = 1$. From the 
probability axioms we have $\prob{\Omega} = 1$. 
Also $\setcomp{\Omega} = \emptyset$. Thus we have $\prob{\emptyset} = 0$.
\end{nnprf}

\begin{lem}
For disjoint events $A$, $B$, $C$, 
$\prob{\threecup{A}{B}{C}} = \prob{A} + \prob{B} + \prob{C}$.
\end{lem}
\begin{nnprf}
The events $\twocup{A}{B}$ and $C$ are disjoint and hence:
\begin{align*}
\prob{\threecup{A}{B}{C}} & = \prob{\twocup{(\twocup{A}{B})}{C}} \\
                          & = \prob{\twocup{A}{B}} + \prob{C} \\
                          & = \prob{A} + \prob{B} + \prob{C}.
\end{align*}
\end{nnprf}
 

\begin{lem}
If events $A_1, A_2, \ldots, A_k$ are disjoint and $k$ is finite, we have that 
$\prob{\fourcup{A_1}{A_2}{\ldots}{A_k}} = 
\prob{A_1} + \prob{A_2} + \ldots + \prob{A_k}$.
\end{lem}
\begin{nnprf}
The proof for the case of three events is extended to prove this lemma. 
\end{nnprf}

\begin{lem}
For the event $S = \{ s_1, s_2, \ldots, s_k\}$,
$\prob{\{ s_1, s_2, \dots, s_k\}} = \prob{\sngltn{s_1}} + 
\prob{\sngltn{s_2}} + \ldots + \prob{\sngltn{s_k}}$.
\end{lem}
\begin{nnprf}
The elements in the set $S$ are mutually exclusive (disjoint). Hence:
\begin{align*}
\prob{\{ s_1, s_2, \dots, s_k\}} & = 
\prob{\fourcup{\sngltn{s_1}}{\sngltn{s_2}}{\ldots}{\sngltn{s_k}}} \\
& = \prob{\sngltn{s_1}} + \prob{\sngltn{s_2}} + 
    \ldots + \prob{\sngltn{s_k}}~\mbox{(finite additivity)}.
\end{align*} 
We abuse notation to write $\prob{\sngltn{s_i}}$ as $\prob{s_i}$.
\end{nnprf}

\begin{lem}
If $A \subset B$, then $\prob{A} \leq \prob{B}$.
\end{lem}
\begin{nnprf}
We can write the event $B$ as the union of two disjoint events: 
$\twocup{A}{\twocap{B}{\setcomp{A}}}$. Using the additivity axiom we have
$\prob{B} = \prob{A} + \prob{\twocap{B}{\setcomp{A}}}$. Non-negativity of 
probabilities implies $\prob{A} \leq \prob{B}$. 
\end{nnprf}

\begin{lem}
$\prob{\twocup{A}{B}} = \prob{A} + \prob{B} - \prob{\twocap{A}{B}}$.
\end{lem}
\begin{nnprf}
We will express the event $\twocup{A}{B}$ as the union of the disjoint events: 
$\twocap{A}{\setcomp{B}}$, $\twocap{A}{B}$ and $\twocap{B}{\setcomp{A}}$. Let 
us denote the probability of these events to be $a$, $b$ and $c$ respectively. 
Using this notation we have for the right hand side:
\begin{equation*}
(a + b) + (b + c) - b = a + b + c.
\end{equation*}
Using the additivity axiom the right hand side is also $a + b + c$. 
\end{nnprf}

\begin{lem}
$\prob{\twocup{A}{B}} \leq \prob{A} + \prob{B}$
\end{lem}
\begin{nnprf}
Non-negativity of probabilities means $\prob{\twocup{A}{B}} \geq 0$. Using this 
in the result 
$\prob{\twocup{A}{B}} = \prob{A} + \prob{B} - \prob{\twocap{A}{B}}$, yields the 
required inequality. This inequality is referred to as the \emph{union bound}.
\end{nnprf}

\begin{lem}
$
\prob{\threecup{A}{B}{C}} = \prob{A} + \prob{\twocap{\setcomp{A}}{B}} + 
\prob{\threecap{\setcomp{A}}{\setcomp{B}}{C}}$.
\end{lem}
\begin{nnprf}
We write $\threecup{A}{B}{C}$ as the union of three disjoint events 
\begin{equation*}
\threecup{A}{B}{C} = 
\threecup{A}{(\twocap{\setcomp{A}}{B})}{(\threecap{\setcomp{A}}{\setcomp{B}}{C})}.
\end{equation*}
Invoking finite additivity property of probabilities proves the desired lemma. 
\end{nnprf}



\section{Discrete uniform probability law}
Let $\Omega$ be a finite sample space with $n$ \emph{equally likely} outcomes. 
Let $A \subset \Omega$ have $k$ elements. Then $\prob{A} = \frac{k}{n}$. This 
choice of probability law where every outcome in the sample space is equally 
likely is called the \textbf{uniform probability law}. Under this probability 
law counting plays an important role in the computation of probabilities.
 
\begin{eg}
Consider the same model of two rolls of a tetrahedral die, with all $16$ 
outcomes equally likely. Find the probability of the following events:
\begin{enumerate}
\item
The value in first roll is strictly larger than Svalue in the second roll.
\item
The sum of the values obtained in the two rolls is an even number.
\end{enumerate}
\end{eg}
\begin{sol}
\hfill
\begin{enumerate}
\item
The event of interest is $\{ (2, 1), (3, 1), (3, 2), (4, 1), (4, 2), (4, 3) \}$. 
We have $6$ outcomes each of which has probability $\frac{1}{16}$. The overall probability is thus $\frac{6}{16}$.
\item
The event of interest is $\{ (1, 1), (1, 3), (3, 1), (2, 2), (2, 4), (4, 2), 
(3, 3), (4, 4) \}$. We have $8$ outcomes each of which has probability 
$\frac{1}{16}$. The overall probability is thus $\frac{8}{16}$.
\end{enumerate}
\end{sol}


\section{Continuous uniform probability law}
For continuous sample space, the uniform probability law is the same as saying 
that the probability of an event (subset) is equal to the ``area" of the subset.
Another line.
\end{document}
